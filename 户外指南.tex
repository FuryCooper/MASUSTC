\documentclass[UTF8]{ctexbook}
\usepackage{abstract}

\title{户外运动指南}
\author{南科大川衡社资料部}
\date{\today}

\begin{document}
\maketitle

\begin{abstract}
   本文档用于南科大川衡社交流学习户外知识使用,部分纰漏欢迎各位讨论交流,传达给资料部校正。
    也欢迎友校社团同学交流心得传授经验。
\end{abstract}

\tableofcontents

\chapter{户外基本知识}
\section{训练}

\section{露营}
\subsection{户外徒步露营选址要注意些什么}
\begin{enumerate}
    \item 安全选址
    \begin{enumerate}
        \item 近水:露营休息离不开水,近是选择营地的第一要素。因此,在选择营地时应选择靠近溪流、湖潭、河流边,以便取水。但也不能将营地扎在河滩上,有些河流上游有发电厂,在蓄水期间河滩宽、水流小,一旦放水时将涨满河滩,包括一些溪流,平时小,一旦下暴雨,都有可能发大水或山洪暴发,一定要注意防范这种问题,尤其在雨季及山洪多发区。
        \item 背风:在野外扎营,不能不考虑背风问题,尤其是在一些山谷、河滩上,应要选择一处背风的地方扎营。还有注意帐篷门的朝向不要迎着风。背风同时也是考虑用火安全与方便。
        \item 远崖:扎营时不能将营地扎在悬崖下面,这样很危险,一旦山上刮大风时,有可能将石头等物刮下,造成伤亡事故。
        \item 近村:营地靠近村庄有什么急事可以向村民求救,在没有柴禾、蔬菜、粮食等情况时就更为重要。近村的同时也是近路,即接近道路,方便队伍的行动和转移。
        \item 近村:营地靠近村庄有什么急事可以向村民求救,在没有柴禾、蔬菜、粮食等情况时就更为重要。近村的同时也是近路,即接近道路,方便队伍的行动和转移。
        \item 背阴:如果是一个需要居住两天以上的营地,在好天气情况下应当选择一处背阴的地方扎营,如在大树下面及山的北面,最好是朝照太阳,而不是夕照太阳。这样,如果在白天休息,帐篷里就不会太闷热。
        \item 防雷:在雨季或多雷电区,营地绝不能扎在高地上、高树下或比较孤立的平地上。那样很容易招至雷击。
        \item 如果你是在大型动物的栖息地扎营,那么晚上很有可能遇袭。
        \item 如果是在冬季或春季的雪山上建立营地,你要确保营地不在雪崩区域。
        \item 如果在海边、沙滩上扎营,要确保营地的位置不受涨潮的影响。
    \end{enumerate}
    \item 合理的规划
    一个齐备的营地应分帐篷宿营区,用火区,就餐区,娱乐区,用水区(盥洗),卫生区等区域。用火区应在下风处,以防火星烧破帐篷。就餐区应就近用火区,以便烧饭做菜及就餐。活动及其娱乐区应就餐区的下风处,以防活动的灰尘污染餐具等物。卫生区同样应在活动区的下风处。用水区应在溪流及其河流上分别上下两段,上段为食用饮水区,下段为生活用水区。
    \item 建设帐篷
    建设帐篷露营区:如有数个帐篷组成的帐篷营地区,在布置帐篷时,应注意:帐篷门都向一个方向开、并排布置。帐篷之间应保持一定的间距。 都向一个方向开、并排布置。帐篷之间应保持一定的间距。
帐篷座落於舒适的位置可以增加约30度的温度,营地不要选择於溪底,因为此处是冷空气聚集处,且不要扎营於山脊棱线,选择背风面或森林处或是运用露宿袋或挖雪洞的方式。
\end{enumerate}

\section{安全急救}
\subsection{准备一个急救包}
淘宝上现成购买一个急救包有不少好处,它更加便宜,也包含了各种各样的医疗用品。但是自己DIY一个急救包会有更多的好处,亦可以根据你要进行的户外活动进行搭配,让保障变得更全面,也可以适当减少一些重量。
以下是一些搭配的小技巧:

一、开始之前

首先,最好的急救箱就是你的急救知识。
如果你打算前往偏远地区度过一段较长的时间,最好提前参加或学习一个野外急救课程。
如果你打算进行短期徒步,你也应该学会最基本的医疗用品使用方法。
从各个网站或者书籍中进行学习,有条件的话,参与到当地的一些急救课程中。

二、选择一个稳妥的包

急救包是你在野外遇险的最后保障,因此无论何时都要好好地保护好它。
一个耐用的户外防水袋是不错的选择,你可以根据需要选择不同的大小,用来存放你的医疗用品。
使用透明的密封塑料袋来装不同品类的医疗用品,以快速找到你需要的东西。
用黑色马克笔在塑料袋上写上名称,以便在不便移动时,指导同伴进行救援。

三、选择合适的医疗用品

一旦你准备好的急救包和密封塑料袋,你就可以按照使用习惯来对其进行分类了。
常用的一般分类有:伤口护理,药物,绷带和其他项目。
以下是这些分类的常见医疗用品,如果你没有太多头绪的话,可以参考以下的清单。
\begin{enumerate}
    \item 伤口护理:\\
    手套:防止双手感染伤口。\\
    医用酒精棉片或其他消毒片:便携的户外消毒医疗用品。\\
    抗生素软膏:抑制继发性皮肤感染。\\
    医用不战纱布垫:清理伤口,伤口敷药。\\
    水泡贴:处理徒步常发的水泡症状。\\
    止血海绵:快速止血抑菌抗感染。
    \item 常用药物:\\
    布洛芬:对乙酰氨基酚(扑热息痛):用于阵痛及治疗发烧。\\
    抗过敏药:减轻过敏反应。\\
    电解质片:用于脱水。
    \item 外用绷带:\\
    不同吃尺寸的纱布绷带包扎伤口。\\
    止血贴:处理小伤口。
    \item 其他项目:\\
    镊子:处理伤口或医疗用品。\\
    小剪刀:处理医疗用品。\\
    针筒:冲洗伤口。\\
    急救信息单:记录个人信息。\\
    运动舒缓喷雾:紧急治疗肌肉关节不适即扭伤等情况。
    \item 备选:\\
    如果你要进行长途徒步,或者在风险更大的情况下,考虑添加以下医疗用品:\\
    紧急夹板:固定骨折或扭伤处。\\
    温度计:测量身体温度。\\
    弹力绷带:固定扭伤关节。\\
    三角绷带:多功能包扎医疗用品。\\
    医用洗眼液:清理进入眼中的异物。\\
    止泻药:治疗腹泻等腹部问题。\\
    CPR急救面罩:人工心脏复苏用品。\\
    救生毯:紧急情况下的庇护所,多种用途。\\
    定位器:表明自身位置。
\end{enumerate}
为了可以更好地使用你的急救包,请根据你的具体活动来选择合适的用品。
如果你要进行攀岩,请带上额外的医用胶带和治疗手指受伤的药品。
如果目的地经常有明显的过敏源出现,请带上抗过敏的药品。 
在每次出发之前,都选择好需要的医疗用品,并保证它们没有缺货,并养成习惯。

\chapter{山岳图册}

\end{document}